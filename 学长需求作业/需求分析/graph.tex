\section{hgz}
描述浏览保险信息的需求相关的实体的属性和交互。
[浏览保险信息类图]
该图由用户、保险列表、保险详细信息、数据存储4类共同组成。数据存储作为数据提供者,可以获取一般保险列表或者推荐保险列表。保险列表除了将一些保险信息作为属性,还会有一些额外属性,并且有搜索筛选功能。保险详细信息之间可以进行对比。

描述浏览保险信息需求中的实体交互顺序
[浏览保险信息顺序图]
在图中,用户先向系统获取保险列表(通过请求一般列表或者推荐列表),可选地对列表进行关键词搜索或者条件筛选保险列表。查看保险详细信息之后,可以可选地对感兴趣的保险信息的属性进行对比。

描述浏览保险信息需求中所经过的状态和转移。
[浏览保险信息状态图]
状态主要有就绪、列表就绪、展示详细信息、展示数据对比四种。在预想中列表就绪是一个主干状态,要浏览详细信息或者进行数据对比都需要经过列表就绪状态。数据对比状态需要经过详细信息状态,与顺序图的交互顺序一致。

描述浏览帖子信息的需求相关的实体的属性和交互
[浏览帖子信息类图]
该图与浏览保险信息类图类似,由用户、帖子列表、帖子和数据存储4类共同组成。除了与浏览保险信息类图相似的部分,用户还可以进行发帖或者回帖。

描述浏览帖子信息需求中的实体交互顺序
[浏览帖子信息顺序图]
在图中,用户先向系统获取帖子列表(通过请求一般列表或者推荐列表),可选地对列表进行关键词搜索来筛选保险列表。

描述发帖需求中的实体交互顺序
[发帖顺序图]
在该图中,用户与系统交互,通过系统设置帖子的各个属性,最终提交,由系统进行持久化,用户获得持久化的结果。

描述回帖需求中的实体交互顺序
[回帖顺序图]
与发帖顺序图不同的是,系统此时读取已持久化的帖子,准备编辑信息。用户完成编辑后提交,由系统进行持久化,返回结果。因为回帖是与浏览帖子的联系并不紧密,因此两个需求的顺序图分开来画。

描述浏览帖子信息与回帖需求中所经过的状态和转移
[浏览帖子与回帖状态图]
图由就绪、列表就绪、帖子内容和编辑回帖内容4种状态组成,需要注意的是编辑回帖内容的状态需要经过查看帖子内容状态。

描述发帖需求中所经过的状态和转移
[发帖状态图]
发帖状态图较为简单,在编辑结束后,由编辑界面状态转为提交结果状态,进而结束发帖流程。

描述查看产品购买记录需求相关的实体的属性和交互
[查看产品购买记录类图]
该图中有较多的类,其中保险公司、普通用户、管理员、客服都属于用户的范畴,在设计中,用户都可以有查看产品购买记录的操作,但角色细分后这类操作的细节不同,所能看到的产品购买记录范围不同。普通用户指可以看见自己的购买记录,管理员或客服可以看到所有的产品购买记录……数据存储类根据请求发起者的不同身份提供不同的产品购买记录。

描述查看产品购买记录需求相关的实体交互顺序
[查看产品购买记录顺序图]
该顺序图中隐藏了用户的具体身份,抽象为用户,向系统发出请求,接受数据。

描述查看产品购买记录需求中所经过的状态和转移
[查看产品购买记录状态图]
该状态图较为简单,系统就绪,用户发送请求,系统进入展示产品购买记录状态,交互结束。


\section{xyd}
描述用户评价咨询的需求相关的实体的属性和交互。
[用户评价咨询的类图]
该类图包含了用户,用户评价,用户咨询,数据存储四个类。用户类中存有用户自身的信息,以及评价列表和咨询列表。评价的类中包含了评价内容,时间,对应的服务等属性。咨询的类中包含了咨询内容,咨询时间&时长,咨询对象等属性。用户的评价和用户咨询的信息都存储在数据库中。

描述用户评价和咨询需求中的实体交互顺序
[用户评价和咨询顺序图]
在用户评价的过程中,用户先创建一个评价类的对象,并且将其加入用户类的评价列表中。用户编写完评价并且提交之后,将评价更新到数据库持久化存储。系统返回用户提交成功的信息。
在用户咨询的过程中,用户先创建一个咨询类的对象,并且将其加入用户类的咨询列表中。用户结束咨询之后,将咨询信息更新到数据库持久化存储。系统返回用户提交成功的信息。
用户可以对自己的评价列表和咨询列表进行关键词搜索或者筛选查询。

描述用户评价过程中所经过的状态和转移。
[用户评价状态图]
用户的主要状态有就绪,编辑评价,提交评价。状态的顺序和顺序图一致。在提交评价之后看到返回提交成功的信息之后结束。

描述用户咨询客服过程中所经过的状态和转移。
[用户咨询客服状态图]
用户的主要状态有就绪,请求咨询,匹配空闲客服,问题解决。状态的顺序和顺序图一致。在咨询客服的过程中,用户不需要选择具体某位客服,而是随机进行空闲匹配。问题解决之后,将咨询的记录存在数据库中,最终结束。

描述用户咨询领域专家过程中所经过的状态和转移。
[用户咨询领域专家状态图]
用户的主要状态有就绪,请求咨询,专家列表就绪,预约专家,正式咨询这几个状态。用户需要先查看专家列表选择合适的专家,用户在查看列表的时候,可以选择查看专家的详细信息。选择合适的专家之后再进行预约咨询,咨询结束后将咨询记录存在数据库中,最终结束。

描述账号管理相关的实体的属性和交互。
[账号管理的类图]
该类图包含了管理员,用户列表和数据存储三个类和实体。管理员的类中存储了管理员自身的信息,如姓名、工号等。用户列表存所有的用户信息,和条目总数,管理员可以对用户列表进行增删改查的操作。修改完成之后再更新到数据库中。

描述用账号管理实体交互顺序
[账号管理顺序图]
管理员可以从数据库里直接筛选查询用户。如果要更新数据库中的用户信息,需要先获取用户列表这一实体,在用户列表内修改之后再持久化到系统中。

描述账号管理状态和转移。
[账号管理状态图]
主要状态有就绪,列表就绪,提交结果。列表就绪之后,管理员可以对列表进行增删改查的操作,然后提交到数据库中持久化存储。状态的顺序和顺序图一致。