%!TEX program = xelatex
\documentclass[a4paper]{ctexart}

\usepackage{listings} 
\usepackage{geometry}
\usepackage{booktabs}
\usepackage{graphicx}
\usepackage{tabularx}
\usepackage{multirow}
\usepackage{enumitem}
\usepackage[bottom]{footmisc}

\renewcommand{\multirowsetup}{\centering}

\geometry{
    left=23mm,
    right=23mm,
    top=23mm,
    bottom=23mm,
}

\setlength{\parskip}{0.5em}

\title{CodeOcean 竞品分析文档}
\author{
  曾少勋 171250603\\
  刘洪禹 171840773\\
  黄国钊 171250530\\
  夏雨笛 171250011\\
}
\date{\today}

\begin{document}

\maketitle

\begin{abstract}
  本项目为 CodeOcean 小组在 2020 年春季学期《需求与商业模式创新》课程中大作业设计的同名项目,此文档为其竞品分析文档。
\end{abstract}

\tableofcontents

\newpage

\setlength{\parskip}{1em}


\section{竞品分析}

\subsection{慕课(MOOC中国)}
网络线上教育平台,采用直播和录播的形式,让用户学员可以在线就学到大学老师开设的各类课程,时间上选择更为灵活。同时,为了改善教学成果,平台还兼备作业、练习、考试和认证功能,让线上教育尽可能贴近线下教育的完整性和有效性。 

从市场角度来看,慕课迎合和目前在线教育的趋势,也符合用户学员对于时间和课程品质的需求和诉求。但是慕课并没有很好的用户粘性,这一类网课平台往往教学的周期较长,可能需要好几个月的持续学习,而对于目前的很多用户来说,他们希望能尽快就获取到核心知识,或者说需求的知识不仅仅局限于大学内的知识,更多的是偏向于实践和工作类的经验,或是对于当前流行的工具和框架的学习,而慕课主要还是聚焦于大学里面的基础知识课程。因此我们小组构思的在线网课功能更加聚焦于时间短、时效性强、实践要求高的精品课程,这样对于用户群体更加有针对性,聚焦于刚刚踏入工作,或是对当前主流框架和编程工具的使用有迫切需要的群体而不是学习基础知识的大学生。 

从行业影响力来看,慕课主要的合作伙伴来自于各个大学的老师,这对于在校的大学生有吸引力,而我们构思的平台的课程开设者主要是来自于企业和开源社区的工程师,这一类工程师的存在本身就会为平台设立一个良好的品牌形象。由于我们的教学内容整体是偏向于企业实战和当下流行的框架和工具,因此这样的品牌形象无疑有利于我们在网课平台脱颖而出。 

从关键趋势来看,线上教育无疑是未来教育的主流和趋势,慕课的网课更加偏向于对传统线下大学教育的一个扩展和延伸,而我们构思的网课平台实质上是针对于“后大学教育”,即我们聚焦的知识领域是和行业发展紧密相关的。往往刚刚步入工作的大学生会面临一个疑惑:在大学里面学习的知识在工作中似乎很不够用。我们构思的平台正式致力于解决这样一个问题。而随着未来的行业发展越来越快,新的框架和编程工具会层出不穷,这样聚焦于这一领域的教育需求也会更加旺盛。

从宏观经济来看,全球化的趋势越来越明显,慕课可以整合各种的教育资源;同样,我们平台也可以整合各种各样的企业资源。更进一步,我们可以做到大学和企业的无缝对接,让刚步入工作的大学生能更快地适应工作上的要求。 


\subsection{Stack Overflow}
Stack Overflow 并非独立公司,而是 Stack Exchange Network 的一个专注编程领域问答的子站,是世界上最著名的编程问答网站之一。Stack Overflow 之所以能在众多问答网站中脱颖而出,关键的一点便在于其“声望值”系统,用户提供的优质回答越多,得到更多的好评,“声望值”也就越高,而给出糟糕的回答,或者提出可以轻易搜索到的问题,都会被其他用户给差评,从而造成“声望值”的降低。从网站内的许多功能均需达到一定的“声望值”才可以使用,这使得 Stack Overflow 达到了一定程度上的“自治”。

从关键趋势来看,云与容器技术目前已经比较成熟,部分代码沙箱平台已经出现,但这些平台往往都只针对一种特定的语言,而且尚未与 Stack Overflow 进行集成,用户提问时需要先找到相应的平台,创建仓库,提交代码,再将地址写到 Stack Overflow 的问题内容中,这个过程较为复杂,对用户的水平要求也较高。而我们的平台在设计之初就已集成代码沙箱功能,用户提问时即可自动创建沙箱,回答者在回答时也无需下载代码、配置环境,直接在网页中便可修改代码并实时看到结果,极大地提升了沟通的效率。

从市场影响力来看,目前编程行业十分火热,许多人都希望能学习一定的编程知识,各种“幼儿编程”、“技术培训班”层出不穷,而国内却始终没有一个良好的编程学习平台。Stack Overflow 面向的主要是程序员、开发者,初学者提出的简单问题常常会被直接 close。而我们的平台则应把握住大量新生的编程初学者,吸引这一部分人到问答平台进行讨论。同时,作为对平台的课程内容的补充,用户对于课程的问题也可以直接在平台上得到回答。

从行业影响力来看,Stack Overflow 极高的知名度使得其“声望值”不再只是网站中的数字,而成为了公司招聘中的“硬实力”,这使得用户有了更强的动力去提升自己的“声望值”,为网站贡献更多的回答。我们的平台也会记录用户的学习、贡献等数据,并在用户同意的情况下将这些数据共享给企业,为企业在招聘时提供一定的参考。

从宏观经济影响来看,Stack Overflow 是公益性质的平台,主要依靠广告与企业私有化部署来盈利。而我们的平台还拥有企业、学校、开源组织、云服务商等多种合作方,我们作为中间平台,可以使得这些实体之间产生交流,共同获利。


\subsection{CSDN}
CSDN是中国专业IT社区。他在很多平台都提供了各种各样功能齐全的服务,旗下拥有:专业的中文IT技术社区: CSDN.NET;移动端开发者专属APP: CSDN APP、CSDN学院APP;新媒体矩阵微信公众号:CSDN资讯、GitChat等;IT技术培训学习平台: CSDN学院;技术知识移动社区: GitChat;人工智能新社区: TinyMind;权威IT技术内容平台:《程序员》+ GitChat;IT人力资源服务:科锐福克斯;IT技术管理者平台:CTO俱乐部。

从关键趋势来看,IT类问答技术社区目前已经比较成熟,吸引了许多用户利用问答技术社区中的帖子学习编程,而且也有不少人热衷于写博客分享自己的学习经验,CSDN也是如此。但是CSDN上有的帖子出现了明显的抄袭转载的嫌疑,不尊重原创作者的版权,无法很好维护与优质内容创作者的双赢关系,从而口碑走向下坡路,优质作者流失。而我们的平台会尽最大能力保证原创作者的利益,优质的解答或者文档会对作者进行打赏。减少抄袭博客的现象,解答发布前进行查重检测。

从市场角度来看,目前从事IT行业的人和IT相关专业的学生越来越多,当然这些用户在工作和学习的时候遇到的问题也都不尽相同。现在的市场中并不缺乏解决用户共性问题的平台,这类共性问题往往和自身的环境关系不大,解决方法都差不多。但是能帮用户解决一些可能由于自身编程环境引起的特性的问题的平台非常少见。将报错信息放在CSDN里搜索,可是搜索出来的解决方案往往很不一样,也很难应用到用户所在的环境当中。而我们的平台提供了解答小哥,可以一对一解决用户的问题。而且利用docker解决了提问者和回答者环境不同的问题。

从行业影响力来看,CSDN涉及各种不同的平台服务:问答社区,平台公众号,学院精品课程等。各种服务的受众叠加起来就是庞大的客户群体,提高了平台的影响力。我们的平台也是如此,提供了网课教学,答案外卖,问答平台和文档中心多种服务,无形之中扩大了用户面,增大规模。

从宏观经济来看,CSDN主要是通过知识付费来盈利,将一些资源和功能设成付费使用。同样我们的平台的部分网课资源,解答服务也需要付费使用。而且如今IT行业蒸蒸日上,会有不少公司愿意赞助IT相关的产业。


\subsection{网易公开课}
网易公开课是一个公开的免费课程平台,用户可以在线免费观看来自于哈佛大学等世界级名校的公开课课程,可汗学院,TED等教育性组织的精彩视频,内容涵盖人文、社会、艺术、科学、金融等领域。并且大部分课程都配有中文字幕。

从市场角度来看,现在我们正处在一个知识付费的时代,很多知识性资源都要付费才能获得。网易公开课这样的免费平台简直就是一枝独秀。但是这上面的课程往往都以大学的课程为主,理论知识较多,实践经验的课程比较少,而且教学周期比较长。但我们平台还推出了长度在五分钟左右的短视频小课,帮助用户迅速学会 一个小知识点,或者指导用户配置环境。能吸引一些不仅仅是以系统学习为导向的用户,更是一些以问题为导向的用户,扩大了市场上的用户范围。

从关键趋势来看,IT行业的从事者或者学生更需要集各种功能为一体的平台,而不是功能单一的平台。对于IT从事者或者学生来说,网易公开课仅仅提供的是网课教学资源,这远远不能满足他们在工作学习中的其他需求。但是我们平台为用户提供了多种功能,网课教学,答案外卖,问答平台和文档中心多种服务,用户只用一个平台就可以满足需求,一键解决,非常便捷。

从行业影响力来看,网易公开课的合作伙伴来源于各个名校的老师,国外知名演讲者。但是由于很多国外的课程都是英文的,增加了用户理解的难度,对于英语不好的人来说并不友好。但是我们平台的大部分的合作都是中国高校和中国企业,更容易树立国产的品牌形象,吸引更多的国内IT行业从事者和学生。

从宏观经济来看,网易公开课是一个纯公益性质的平台,目的是为了攒口碑,为网易的其他产品导流量。而我们的平台能让高校企业,或者解答小哥,文档贡献者共同获利。


\subsection{极客学院}
极客学院是一个IT行业方向的在线教育平台,其为用户提供了在广度和深度上都有保障的课程资源:从一门语言的简单学习到云平台之类的技能深造,从视频课到实战训练营或者个人定制化的学习计划设计,它都为客户提供了多种多样的选择。客户可以通过为单门课程付费或者成为会员享受多种课程的方式进行学习。

就市场影响因素来说,随着互联网的发展,中国从事IT行业的人在迅速增多,因此渐渐也产生了通过互联网进行IT行业方向学习的需求。但互联网资源的分散性也使得很多人希望能有整合互联网教学资源的平台,因此不少在线教育平台开始出现,极客学院抓住了在线教育平台的发展浪头,目标一直定位在职业在线教育方向,积极寻求与企业高校等组织的合作来促进职业在线教育的推广。其使用在广度和深度上都十分丰富的资源来吸引顾客。我们的平台定位是轻松的教学社区,即使是对编程只限于兴趣的人也是我们的用户。

从行业影响因素来看,现在在线教育行业还在初期开发之中,各家企业都在积极探索立下脚跟发展自己的道路,极客学院也在积极开拓其垂直服务链,目前极客学院已经推出了实战训练营这种有一定实践成分的课程,并还在积极向下探索,争取与企业对接,囊括用户从接受教学到就职的整条路线。一般来说在线教育平台会与云平台和IT相关企业和院校产生依赖关系,和这些合作者保持好的关系是企业发展的重要一环。我们的平台并不引导用户至就业,而是以创造知识共享的平台为主,提供多种方式来解决用户在编程上的疑问。

从关键趋势来看,面对互联网的发展,国家对于在线教育平台也有一定的支持意向,特别是此次疫情之后,在线教育平台生态的发展可能会再进一步,但一般来说是通识教育平台,对于职业在线教育平台的方向,空白还是比较多;并且整个互联网的产品呈现多寡头垄断的趋势,极客学院积极寻求与大公司的合作以发展自身。我们的平台在课程设置上的竞争力可能存在不够强的问题,但我们的价值主张允许我们设置一些短课程,能补充一些竞争力。

从宏观经济来说,虽然中国的经济发展放缓,但整体还是稳定向好;云平台有大公司开展业务,基础设施稳定;面向IT职业培养的教学人才培养体系还比较混乱,争夺优秀的人才需要付出一定的成本,短时间内成本大幅下降的可能性不大。我们对于优秀教学人才的依赖性不如极客学院那么强,当下的宏观经济环境对于我们问答平台的建设影响不大,因此整体的建设成本较为平均。


\end{document}